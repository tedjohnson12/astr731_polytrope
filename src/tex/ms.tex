% Define document class
\documentclass[twocolumn]{aastex631}
\usepackage{showyourwork}
\usepackage{listings}
\usepackage{tabularx}
% Begin!
\begin{document}

% Title
\title{Stellar Interiors Numerical Assignment}

% Author list
\author{Ted Johnson}


\section{Introduction}
\label{sec:intro}

We will construct a polytropic model of a stellar interior using
polytropic index $0\le n \le 4$. Following Section 7.2.2 of
\citet{textbook}, we cast the Lane-Emden equation into two
first-order differential equations before numerically solving
with a fourth-order Runge-Kutta integrator.

The integrator is implemented in native Python in the repository
linked at the top of the page (see \texttt{polysolver/}). It is
also re-implemented in Rust (\texttt{rust/}). In addition to
learning about numerical solvers and polytropes, this project
allowed me to experiment with Rust-Python bindings and with
\showyourwork \citep{Luger2021}, which makes it possible for
this document (and all the source code associated
with it) to be open-source and completely
reproducible.

Following \citet{textbook} we will focus on the profile $\theta_n(\xi)$
as well as the location of the surface $\xi_1$, the derivative
of $\theta_n$ at the surface $\theta_n'(\xi_1)$, and the central
density in units of the bulk density $\rho_c/<\rho>$.

Section \ref{sec:methods} describes the code and it's outputs,
Section \ref{sec:res} examines the effects of numerical resolution
on the results, ...

%describe sections

\section{Methods}
\label{sec:methods}
Most of the work is done by the Python function
\begin{lstlisting}[language=Python]
    polysolver.solve(
        x_int,n,h,
        max_iter=1000,impl='rust'
    )->x,y,z
\end{lstlisting}

The details of the code's backend can be easily read from
the source code itself, but it essentially starts with some
values $x$, $y$, and $z$ and then computes the next value for
each using a 4th-order Runge-Kutta integrator. It stops when
the condition $y \le 0$ is met or after \texttt{max\_iter} iterations.

These three variables are defined by \citet{textbook} as
\[ x = \xi\]
\[y = \theta_n\]
\[z = \frac{\text{d}\theta_n}{\text{d}\xi}\]

We also define $\xi_1$ as the value of $\xi$ when $\theta_n$ is
zero (i.e. the surface).

\texttt{x\_init} is the initial value of $x$ and must be set close to zero.
However, it cannot be zero because at zero the derivative of $z$ goes to infinity.
Unless stated otherwise, we set \texttt{x\_init} to $10^{-20}$.

\texttt{n} is the polytropic index of the model. We explore
$0\le n \le 4$.

\texttt{h} is the step size. This is an important parameter that we will
explore below. Ideally this parameter should be less than
the pressure scale height at all points.

\texttt{max\_iter} and \texttt{impl} are optional parameters that do not
effect the mathematics of the model. \texttt{max\_iter} prevents
an infinite loop, and should be set high enough that the
condition $y\le 0$ is met. \texttt{impl} is the solver
implementation that will be used. The code has been implemented
in both Python and Rust.


\section{Resolution Study}
\label{sec:res}

\begin{figure}
    \script{res_n0.py}
    \begin{centering}
        \includegraphics[width=0.45\textwidth]{figures/res_n0.pdf}
        \caption{Polytropic curve for $n=0$ for various
        values of $h$. An analytic solution is used as
        the ground truth. Note that the residuals are highest
        near the boundaries.}
        \label{fig:res_n0}
    \end{centering}
\end{figure}

\begin{figure}
    \script{res_n1.py}
    \begin{centering}
        \includegraphics[width=0.45\textwidth]{figures/res_n1.pdf}
        \caption{Polytropic curve for $n=1$ for various
        values of $h$. An analytic solution is used as
        the ground truth. Note the local minimum at
        $\xi\sim \pi/2$. There is an inflection point here,
        and our numerical solution crosses the true value.
        The location of this minimum approaches the inflection
        point as $h$ approaches zero.}
        \label{fig:res_n1}
    \end{centering}
\end{figure}

\begin{figure}
    \script{res_xi1.py}
    \begin{centering}
        \includegraphics[width=0.45\textwidth]{figures/res_xi1.pdf}
        \caption{Deviations of $\xi_1$ from the true value as a
        function of $h$. Polytropes without analytic solutions
        quickly converge as $h$ approaches zero, while the
        $n=0$ and $n=1$ solutions continue to imporove indefinitely.
        It is expected that these values would eventually
        be limited by machine precision or \texttt{x\_init}.
        They y-axis is defined as
        $\Delta q = \frac{|q-q_{\rm true}|}{q_{\rm true}}$.
        }
        \label{fig:res_xi1}
    \end{centering}
\end{figure}

\begin{figure}
    \script{res_theta_prime.py}
    \begin{centering}
        \includegraphics[width=0.45\textwidth]{figures/res_thetaprime.pdf}
        \caption{Same as Figure \ref{fig:res_xi1}, but for
        $\theta_n'(\xi_1)$.}
        \label{fig:res_thetaprime}
    \end{centering}
\end{figure}

\begin{figure}
    \script{res_rho.py}
    \begin{centering}
        \includegraphics[width=0.45\textwidth]{figures/res_rho.pdf}
        \caption{Same as Figure \ref{fig:res_xi1}, but for
        $\rho_c/<\rho>$. The curve for $n=1$ is mostly covered
        by the $n=0$ curve.}
        \label{fig:res_rho}
    \end{centering}
\end{figure}

The step size parameter $h$ determines the resolution of our integration
in units of $\xi$. A more sophisticated algorithm might choose a variable
step size to improve accuracy, but we hold this parameter constant
for each integration.

A major flaw in our choice of constant $h$ is that the likelyhood that
the program terminates when $\theta_n=0$ is very small. Except when $h$
and \texttt{x\_init} are chosen very carefully, $\xi_1$ will end up
between grid points -- producing additional error for the value
of our solution at the surface.

To mitigate this effect we can modify our equation for $z'$
\[
    z' =
    \begin{cases}
        -y^n - \frac{2}{x}z & \text{if}~y\ge0 \\
        (-y)^n - \frac{2}{x}z & \text{if}~y<0
    \end{cases}
\]
This way $z'$ is continuous and real as $y$ crosses zero. We then
record one additional point $\xi>\xi_1$ and use interpolation
to determine the surface value. We choose a cubic spline interpolation
of the final three points for a balance between accuracy and
speed.

Figures \ref{fig:res_n0} and \ref{fig:res_n1} show the
polytropic curve for $n=0$ and $n=1$ for various
values of $h$. The curve shown is a resampled solution from
\texttt{x\_init} to $\xi_{1,\text{measured}}$ using a cubic spline.
The resampling is done so that the value of $\theta_n$ at
the surface can be shown most accuratly.





We can also discuss the effects of $h$ on our measured quantities
$\xi_1$, $\theta_n'(\xi_1)$, and $\rho_c/<\rho>$. \citet{textbook}
gives values for these quantities for $n=0,1,1.5,2,3,4$ in Table 7.1.
Deviations of our numerical results from these given quantities fall
into a few regimes:

\begin{enumerate}
    \item Dominated by interpolation errors. This is our uncertainty
    due to $h$. \label{it:interp}
    \item Dominated by the finite precision of the true value. Once
    within the precision of the value given by \citet{textbook},
    improving the resolution no longer yields a better agreement.
    \label{it:prec}
    \item Dominated by the value of \texttt{x\_init}. Since we
    cannot start our integration at $\xi=0$, our numerical solution
    will converge differently depending on the location of this boundary.
    \label{it:init}
\end{enumerate}

We expect Regime \ref{it:interp} to be the main source of error for
large $h$. Figures \ref{fig:res_xi1}, \ref{fig:res_thetaprime},
and \ref{fig:res_rho} show that at large $h$ the residuals
are indeed a function of $h$, but for the case of general $n$ our
model converges when $h$ is sufficiently small. The fact that we
do not see convergence for $n=0$ and $n=1$ suggests that the convergence
of residuals in the general case is due completely to finite knowledge
of the true value (i.e. Regime \ref{it:prec}). If it were the case of
Regime \ref{it:init} then the solutions that are known analytically
would also converge to some near-true value and we would see the same behavior
for all $n$.



Herein we choose a value of $h = 10^{-3}$ based on Figures \ref{fig:res_xi1},
\ref{fig:res_thetaprime}, and \ref{fig:res_rho}.

\subsection{The effects of \texttt{x\_init}}
\label{subsec:init}

\begin{figure}
    \script{xinit_xi1.py}
    \begin{centering}
        \includegraphics[width=0.45\textwidth]{figures/xinit_xi1.pdf}
        \caption{Deviations of $\xi_1$ from the true value as a
        function of \texttt{x\_init}. As expected, when \texttt{x\_init}
        is large $\Delta \xi_1$ is a function of \texttt{x\_init},
        and the error is independent of \texttt{x\_init} when it is small.
        This relationship holds for arbitrary $n$, confirming that
        for small $h$ and small \texttt{x\_init} the numerical
        error we measure is dominated by the true value's uncertainty.}
        \label{fig:xinit_xi1}
    \end{centering}
\end{figure}

Let us quickly examine the effects of our lower boundary on our
numerical results. Figure \ref{fig:xinit_xi1} shows the
deviations of $\xi_1$ similar to Figure \ref{fig:res_xi1},
but in this case our independent variable is \texttt{x\_init}.
For every solution there is a value of \texttt{x\_init}
after which there is no improved accuracy. This value appears to be
$~10^{-3}$, but that is also the value we set for $h$.

Figure \ref{fig:xinit_res_xi1} examines the joint effects of 
\texttt{x\_init} and $h$. We do find that for $n=0$ and $n=1$
\texttt{x\_init} should be at a maximum $h$, but for general $n$
we become dominated by Regime \ref{it:prec} when \texttt{x\_init}
is less than $\sim 10^{-2}$. However, unlike $h$, we can choose arbitrarily
small \texttt{x\_init} without sacrificing computational
resources, so we will continue with the value $10^{-20}$.



\section{Results}
\label{sec:results}

\variable{output/tab7.txt}

\begin{figure}
    \script{func_of_n.py}
    \begin{centering}
        \includegraphics[width=0.45\textwidth]{figures/func_of_n.pdf}
        \caption{Quantities $\xi_1$, $-\theta_n'(\xi_1)$,
        and $\rho_c/<\rho$ as a function of polytropic index $n$.
        As $n$ increases our solution flattens out near the surface.
        The radius, however, does not increase by very much, so
        we see a concentration of material neart the core.}
        \label{fig:func_of_n}
    \end{centering}
\end{figure}

Now that we have an established understanding of the errors in our study
we will look at our numerical results for arbitrary $n$. As stated in
Section \ref{sec:res}, we choose a value of $h=10^{-3}$ and set 
\texttt{x\_init} to $10^{-20}$. Table \ref{tab:7} gives these results
in the style of Table 7.1 in \citet{textbook}.



Figure \ref{fig:func_of_n} shows the quantities $\xi_1$, $-\theta_n'(\xi_1)$,
and $\rho_c/<\rho$ as continuous functions of polytropic index $n$.


\appendix

\section{Additional Figures}
\label{sec:figs}[h!]
\begin{figure*}
    \script{xinit_res_xi1.py}
    \begin{centering}
        \includegraphics[width=0.9\textwidth]{figures/xinit_res_xi1.pdf}
        \caption{Deviations of $\xi_1$ from the true value as a function
        of both \texttt{x\_init} and $h$. The dashed line
        marks the curve \texttt{x\_init}$=h$.}
        \label{fig:xinit_res_xi1}
    \end{centering}
\end{figure*}









\bibliography{bib}

\end{document}
