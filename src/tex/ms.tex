% Define document class
\documentclass[twocolumn]{aastex631}
\usepackage{showyourwork}

% Begin!
\begin{document}

% Title
\title{Stellar Interiors Numerical Assignment}

% Author list
\author{Ted Johnson}


\section{Introduction}
\label{sec:intro}

We will construct a polytropic model of a stellar interior using
polytropic index $0\le n \le 4$. Following Section 7.2.2 of
\citet{textbook}, we cast the Lane-Emden equation into two
first-order differential equations before numerically solving
with a fourth-order Runge-Kutta integrator.

The integrator is implemented in native Python in the repository
linked at the top of the page (see \texttt{polysolver/}). It is
also re-implemented in Rust (\texttt{rust/}). In addition to
learning about numerical solvers and polytropes, this project
allowed me to experiment with Rust-Python bindings and with
\showyourwork, which makes it possible for this document (and
all the source code associated with it) to be open source and completely
reproducible.

%describe sections

\section{Methods}
\label{sec:methods}
Most of the work is done by the Python function
\verb|polysolver.solve(x_int,n,h,max_iter=1000,impl='rust')->x,y,z|.
The details of the code's backend can be easily read from
the source code itself, but it essentially starts with some
values $x$, $y$, and $z$ and then computes the next value for
each using a 4th-order Runge-Kutta integrator. It stops when
the condition $y \le 0$ is met or after \texttt{max\_iter} iterations.

These three variables are defined by \citet{textbook} as
\[ x = \xi\]
\[y = \theta_n\]
\[z = \frac{\text{d}theta_n}{\text{d}\xi}\]

\texttt{x\_init} is the initial value of $x$ and must be set close to zero.
However, it cannot be zero because at zero the derivative of $z$ goes to infinity.

\texttt{n} is the polytropic index of the model. We explore
$0\le n \le 4$.

\texttt{h} is the step size. This is an important parameter that we will
explore below. Ideally this parameter should be less than
the pressure scale height at all points.

\texttt{max\_iter} and \texttt{impl} are optional parameters that do not
effect the mathematics of the model. \texttt{max\_iter} prevents
an infinite loop, and should be set high enough that the
condition $y\le 0$ is met. \texttt{impl} is the solver
implementation that will be used. The code has been implemented
in both Python and Rust.


\section{Resolution Study}
\label{sec:res}

We examine the effect of step size on our solutions.

\begin{figure}
    \script{res_n0.py}
    \begin{centering}
        \includegraphics[width=0.5\textwidth]{figures/res_n0.pdf}
        \caption{Polytropic curve for $n=0$ for various
        values of $h$. An analytic solution is used as
        the ground truth. Note that the residuals are highest
        at the boundaries.}
        \label{fig:res_n0}
    \end{centering}
\end{figure}

\begin{figure}
    \script{res_n1.py}
    \begin{centering}
        \includegraphics[width=0.5\textwidth]{figures/res_n1.pdf}
        \caption{Polytropic curve for $n=1$ for various
        values of $h$. An analytic solution is used as
        the ground truth. Note that the residuals less than 0
        at the core and greater than zero at the surface. This
        is due to an inflection point at $\xi=\pi/2$.}
        \label{fig:res_n1}
    \end{centering}
\end{figure}





\bibliography{bib}

\end{document}
